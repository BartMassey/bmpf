\documentclass{article}
\usepackage{spconf,amsmath}
\usepackage{graphicx}
\usepackage{url}

\newcommand{\asgn}{\,\,\leftarrow\,\,}
\newcommand{\newcode}{\\*[0.5\baselineskip]}
\newcommand{\Prob}{\text{Pr}}
\newcommand{\hyperg}{{{{}_2}{F_1}}}
\newcommand{\du}{{\text{d}u}}

\title{Fast Perfect Weighted Resampling}

\name{Bart Massey}
\address{%
  Associate Professor\\
  Computer Science Department\\
  Portland State University\\
  Portland, Oregon~~USA\\
  \url{bart@cs.pdx.edu}}

\begin{document}
\maketitle
%
\begin{abstract}
  We describe an algorithm for perfect weighted-random
  resampling of a population with time complexity $O(m +
  n)$ for resampling $m$ inputs to produce $n$ outputs.
  This algorithm is an incremental improvement over
  standard resampling algorithms.  Our resampling
  algorithm is parallelizable, with linear speedup.
  Linear-time resampling yields notable performance
  improvements in our motivating example of Sequential
  Importance Resampling for Bayesian
  Particle Filtering.
\end{abstract}
%
\begin{keywords}
Monte Carlo methods, state estimation, filtering, tracking filters
\end{keywords}
%
\section{Introduction}

  Bayesian Particle Filtering (BPF)~\cite{bpf} is an exciting
  new methodology for state space tracking and sensor
  fusion.  The bottleneck step in a typical BPF
  implementation is the weighted resampling step known as
  Sequential Importance Resampling: creating a new
  population of ``particles'' from an old population by
  random sampling from the source population according to
  the particle weights.  Consider resampling $m$ inputs to
  produce $n$ outputs.  The na\"ive algorithm has time
  complexity $O(mn)$.  BPF implementations that resample
  using this expensive $O(mn)$ algorithm can afford to
  resample only sporadically; this represents a difficult
  engineering tradeoff between BPF quality and computational
  cost.

  Resampling via binary search is often used in faster BPF
  implementations~\cite{arulampalam02tutorial}.  However,
  since the other steps in a BPF iteration are linear in the
  number of particles, a $O(m + n \log m)$ binary search
  will still be the bottleneck step in such an
  implementation---a factor of 10 slowdown for a 1000-particle
  filter, representing significant extra computation.  The
  memory access patterns of a binary search also interact
  badly with the cache of a modern processor.

  One well-known method that has been used to regain the
  performance lost to resampling is to give up on
  statistical correctness and simply sample at regular
  intervals~\cite{kitagawa}.  In practice, this seems to
  work well, and to run quite quickly.  However, one cannot
  help but be a bit concerned that regular resampling will
  ``go wrong'' in a crucial situation, due to correlations
  between the sampling interval and an unfortunate particle
  set.  Although the idea does not seem to be widely known,
  the resistance of regular resampling to such correlations
  can be improved simply by shuffling the particles before
  resampling.  Since shuffling can be done in $O(n)$ time
  with good constant factors, this is probably a good idea.

  What is really wanted, however, is a resampling algorithm
  with the $O(m + n)$ running time of regular resampling,
  but statistically equivalent to na\"ive resampling.  We
  give such an algorithm, which we believe to be novel.  For
  a simple array representation of the output, simply
  initializing the output will require time $O(n)$.  It
  seems that the input must be scanned at least once just to
  determine the total input weight for normalization.  Thus,
  the running time of our algorithms is apparently optimal.
  We also report on implementations that are
  performance-competitive with regular resampling in a toy
  BPF domain.

\section{Weighted Resampling Algorithms}

  For what follows, assume an array $s$ of $m$ input
  samples, and an output array $s'$ that will hold the $n$
  output samples of the resampling.  Assume further that
  associated with each sample $s_i$ is a weight $w(s_i)$,
  and that the weights have been normalized to sum to 1.
  This can of course be done in time $O(m)$, but typical
  efficient implementations keep a running weight total
  during weight generation, and then normalize their
  sampling range rather than normalizing the weights
  themselves.  We thus discount the normalization cost in
  our analysis.

\subsection{A Na\"ive $O(mn)$ Resampling Algorithm}\label{sec:naive}

  The na\"ive approach to resampling has been re-invented
  many times.  A correct, if inefficient, way to resample is
  via the pseudocode of Figure~\ref{fig:omn}.  The {\em
  sample} procedure selects the first sample such that the
  sum of weights in the input up to and including this
  sample is greater than some index value $\mu$.  The index
  value is chosen in {\em resample} by uniform random
  sampling from the distribution $[0..1]$, with each output
  position being filled in turn.

  \begin{figure}
    \centering
    \begin{minipage}[b]{\linewidth}
      \begin{tabbing}
      XX\=XXXX\=XXXX\=XXXX\=\kill
      {\bf to} {\it sample}$(\mu)$: \\
      \>$t \asgn 0$ \\
      \>{\bf for} $i$ {\bf from} $1$ {\bf to} $m$ {\bf do} \\
      \>\>$t \asgn t + w(s_i)$ \\
      \>\>{\bf if} $t > \mu$ {\bf then} \\
      \>\>\>{\bf return} $s_i$ \newcode
      {\bf to} {\it resample}: \\
      \>{\bf for} $i$ {\bf from} $1$ {\bf to} $n$ {\bf do} \\
      \>\>$\mu \asgn \textit{random-real}([0..1])$ \\
      \>\>$s'_i \asgn sample(\mu)$
      \end{tabbing}
    \end{minipage}
    \caption{Na\"ive Resampling}\label{fig:omn}
  \end{figure}

  Despite its poor performance, the na\"ive algorithm has
  its advantages.  It is easy to verify that it is a perfect
  sampling algorithm.  It is easy to implement, and easy to
  parallelize.  The expected running time is
  $o(\frac{1}{2}mn)$.

  To derive a $O(m + n \log m)$ algorithm from the na\"ive
  algorithm, note that the linear scan of input samples in
  Figure~\ref{fig:omn} can be replaced with a binary
  search.  One way to do this would be to treat the array of
  input samples as a heap.  This heap-based algorithm, not
  shown here for space reasons, does dramatically improve on
  the performance of the na\"ive algorithm without
  sacrificing correctness.

  For some input particle distributions, a further
  constant-factor improvement to both the na\"ive and
  heap-based algorithms can be had by sorting or heapifying,
  respectively, the input particle array so that the largest
  particles are likely to be encountered first in the
  search.  The amortized cost of these operations is small,
  but may be larger than the cost savings in typical
  distributions; the approach also adds a bit to the
  complexity of the implementation.

\subsection{A Merge-based $O(m + n \log n)$ Resampling Algorithm}\label{sec:merge}

  The real problem with the na\"ive
  algorithm is not so much the cost per scan of the input as it is the
  fact that each scan is independent.  It seems a shame not
  to try to do all the work in one scan.
  Let us generate an array $u$ of $n$ variates up-front,
  then sort it.  At this point, a {\em merge} operation, as
  shown in Figure~\ref{fig:merge}, can be used to generate
  all $n$ outputs in a single pass over the $m$ inputs.  The
  merge operation is simple.  Walk the input array once.
  Each time the sum of weights hits the current variate
  $u_i$, output a sample and move to the next variate
  $u_{i+1}$.  The time complexity of the initial sort is
  $O(n \log n)$ and of the merge pass is $O(m + n)$, for a
  total time complexity of $O(m + n \log n)$.

  \begin{figure}
    \centering
    \begin{minipage}[b]{\linewidth}
      \begin{tabbing}
      XX\=XXXX\=XXXX\=XXXX\=\kill
      {\bf to} {\it merge}$(u)$: \\
      \>$j \asgn 1$ \\
      \>$t \asgn u_1$ \\
      \>{\bf for} $i$ {\bf from} $1$ {\bf to} $n$ {\bf do} \\
      \>\>$\mu \asgn u_i$ \\
      \>\>{\bf while} $\mu < t$ {\bf do} \\
      \>\>\>$t \asgn t + w(s_j)$ \\
      \>\>\>$j \asgn j + 1$ \\
      \>\>$s'_i \asgn s_j$
    \end{tabbing}
    \end{minipage}
    \caption{Merge-based Resampling}\label{fig:merge}
  \end{figure}

  Complexity-wise, we seem to have simply moved the $\log$
  factor of the heap-based algorithm from $m$ to $n$,
  replacing an $O(m + n \log m)$ algorithm with an $O(m + n
  \log n)$ one.  However, the new algorithm has an important
  distinction.  The log factor this time comes merely from
  sorting an array of uniform variates.  If we could somehow
  generate the variates in sorted order (at amortized
  constant cost) we could make this approach run in time
  $O(m + n)$.  The next section shows how to achieve this.

\subsection{An Optimal $O(m + n)$ Resampling Algorithm}\label{sec-optimal}

  As discussed in the previous section, if we can generate
  the variates comprising a uniform sample of $n$ values in
  increasing order, we can resample in time $O(m + n)$.
  Assume without loss of generality that our goal is simply
  to generate the first variate in a uniform sample of $n +
  1$ values.  Call the first variate $\mu_0$, the set of
  remaining variates $U$ and note that $|U|=n$.  Now, for
  any given variate $\mu_i \in X$, we have that
    $$ \Prob(\mu_0 < \mu_i) = 1 - \mu_0 $$
  Since this is independently true for each $\mu_i$, we define
    $$p(\mu_0) = \Prob(\forall \mu_i \in U ~.~ \mu_0 < \mu_i) = (1 - \mu_0)^n$$
  
  Thus, if we successively generate $n$ variates $u_i$ drawn from the
  distribution $(1 - \mu_0)^{n-i}$, those variates will be statistically
  indistinguishable from the set of variates produced by generating $n$
  uniform variates and then sorting them.  To generate a
  variate from the target distribution, it is sufficient to
  observe that the likelihood of generating a variate $\mu$
  is given by
    \begin{eqnarray*}
      \mu &=& \frac{\int_{u=0}^{\mu_0}{(1-u)^n \du}}
                   {\int_{u=0}^{1}{(1-u)^n \du}} \\
	  &=& \frac{\left.\frac{-(1-u)^{n+1}}{n+1}\right|_{u=0}^{\mu_0}}
		   {\left.\frac{-(1-u)^{n+1}}{n+1}\right|_{u=0}^{1}} \\
          &=& \frac{\frac{-1}{n+1}\left[(1-\mu_0)^{n+1}-1\right]}
		   {0-\frac{-1}{n+1}} \\
          &=& 1 - (1 - \mu_0)^{n+1}
    \end{eqnarray*}

  However, what we need is $\mu_0$ in terms of
  $\mu$, so we solve
    \begin{eqnarray*}
       \mu &=& 1 - (1 - \mu_0)^{n+1} \\
       (1 - \mu_0)^{n+1} &=& 1 - \mu \\
       \mu_0 &=& 1 - (1 - \mu)^\frac{1}{n+1} \\
       \mu_0 &=& 1 - \mu^\frac{1}{n+1}
    \end{eqnarray*}
  (The last step is permissible because $\mu$ is a
  uniform deviate in the range $0..1$, and therefore
  statistically equivalent to $(1-\mu)$.)

  We now have the formula we need for selecting the first
  deviate from a set of $n$ in increasing order.  To select
  the next deviate, we simply decrease $n$ by $1$, select a
  deviate from the whole range, and then scale and offset it
  to the remaining range.  We repeat this process until
  $n=0$.  (Recall that $|U|=n$, so the last deviate will be
  selected when $n=0$.)  Figure~\ref{fig:deviate} shows
  this process.
  
  \begin{figure}
    \centering
    \begin{minipage}[b]{\linewidth}
      \begin{tabbing}
      XX\=XXXX\=XXXX\=XXXX\=\kill
      {\bf to} {\it randomize}: \\
      \>$u_1 \asgn (1-\mu)^{\frac{1}{n}}$ \\
      \>{\bf for} $i$ {\bf from} $2$ {\bf to} $n$ {\bf do} \\
      \>\>$u_i \asgn u_{i-1} + (1-u_{i-1})(1-\mu)^{\frac{1}{n-i+1}}$
    \end{tabbing}
    \end{minipage}
    \caption{Generating Deviates In Increasing Order}\label{fig:deviate}
  \end{figure}

  We now have the array $u$ of deviates in sorted order that
  we need to feed the {\em merge} algorithm of the previous
  section.  We thus have an $O(m + n)$ algorithm for random
  weighted selection.

\subsection{Faster Generation of Variates}

  The method of directly generating variates described in the

  previous section has a minor limitation: it requires an
  exponentiation per variate.  Even a fast implementation of
  the mathematical function {\tt pow()} on a machine with
  fast floating point is expensive compared to the single
  multiply of regular resampling.

  One possible way around the {\tt pow()} bottleneck is to generate
  random variates with distribution $(1 - x)^n$ by
  Marsaglia and
  Tsang's ``Ziggurat Method'' PRNG~\cite{ziggurat,ziggurat-fixes}.
  Unfortunately, the desired
  distribution is bivariate.  The most direct
  approach would lead to generating $n$ sets of Ziggurat
  tables, which would be prohibitively memory-expensive for
  large $n$.

  When computing $(1 - x)^n$ for large $n$, however, our
  probability expression becomes self-similar, and we can
  accurately approximate the function for larger $n$ using
  the function with smaller $n$.  In fact, this is the
  well-known compound interest problem, yielding an elegant
  limiting approximation.
 $$
\lim_{a \rightarrow \infty}\left(1 - \frac{x}{a}\right)^{an}
  =   \lim_{a \rightarrow \infty}\left(1 + \frac{(-x)}{a}\right)^{an}
  = e^{-xn}$$  This approximation corresponds to a standard 
  linear-time approximate resampling method in which the next sample
  weight is given by an exponentially-distributed
  variate~\cite{carpenter}.  The approximation works well up until near the
  end of the resampling, and is relatively inexpensive if a
  Ziggurat-style exponential generator is used.

  We can get the same performance with perfect resampling,
  though, by modifying a Ziggurat generator for
  $e^{-50\mu_0}$ to accurately compute the desired power by
  tweaking the rejection step.  The efficiency of the
  generator would deteriorate unacceptably in the last 50
  samples, so at that point we just switch to calling {\tt
  pow()} directly.  The resulting resampling implementation
  has performance close to that of regular resampling, but
  with the perfect resampling of the na\"ive method.

  A parallel version of our perfect algorithm is
  straightforward to achieve. Although space precludes a
  detailed description, the basic idea is as follows.  Given
  $p$ processors, each processor generates a variate $v_i$ from the
  distribution $x^i(1-x)^{p-i}$, then binary searches for
  the sample breakpoint corresponding to that variate.  Then
  the processor generates $n/p$ samples in the range
  $v_{i-1}\ldots v_i$ using the perfect resampling
  algorithm.  This achieves a linear speedup while retaining
  statistically correct resampling.

\section{Evaluation}

  We implemented the algorithms described previously in a
  BPF tracker for simulated vehicle navigation.  The
  simulated vehicle has a GPS-like and an IMU-like device
  for navigational purposes; both the device and the vehicle
  model are noisy.  Figures~\ref{fig:track-naive} and
  \ref{fig:track-optimal} show 1000-step actual and estimated vehicle
  tracks from the simulator for the 100-particle case, using
  the na\"ive and optimal algorithms.  Note
  that the quality of tracking of the two algorithms is
  indistinguishable, as expected.

  \begin{figure}
    \begin{minipage}{0.45\linewidth}
    \centering
    \includegraphics[width=0.95\linewidth]{track-naive-100}
    \begin{center}\small (a) Na\"ive Resampling\label{fig:track-naive}\end{center}
    \end{minipage}
    \begin{minipage}{0.45\linewidth}
    \centering
    \includegraphics[width=0.95\linewidth]{track-optimal-100}
    \begin{center}\small (b) Optimal Resampling\label{fig:track-optimal}\end{center}
    \end{minipage}
    \caption{Vehicle Tracking Using BPF}
  \end{figure}

  The important distinction between the algorithms we have
  presented is not quality, but rather runtime.
  Figure~\ref{fig:times} shows the time in seconds for 1000
  iterations of BPF with various resampling algorithms as a
  function of the number of particles tracked / resampled.
  The benchmark machine is an otherwise unloaded Intel Core
  II Duo box at 2.13 GHz with 2GB of memory.

  \begin{figure*}[t]
    \centering
    \includegraphics[width=0.7\linewidth]{times}
    \caption{Runtimes for BPF Resampling Implementations}\label{fig:times}
  \end{figure*}

  As expected, BPF using the na\"ive algorithm becomes
  unusable at larger particle sizes, whereas BPF using the
  optimal algorithm scales linearly.  The heap-based
  algorithms are surprisingly competitive with the optimal
  algorithm even at large particle counts, although the
  distinction is somewhat masked by the mediocre running
  time of the rest of the BPF implementation; resampling is
  not the bottleneck for any of these fast algorithms, as
  the near-zero cost of the regular algorithm without
  shuffling indicates.

  The performance difference between our
  Ziggurat-based implementation of optimal resampling and
  our shuffled implementation of regular resampling is quite
  small.  Given this, we believe that the statistical
  correctness of optimal resampling will make it the best
  choice for many implementations.

\section{Availability}

  Our C implementation of BPF with linear resampling
  described here is freely available under the GPL at
  \url{http://wiki.cs.pdx.edu/bartforge/bmpf}.  It relies on
  our BSD-licensed implementation (partly based on the work
  of others---please see the distribution for attribution)
  of various PRNGs and Ziggurat generators at
  \url{http://wiki.cs.pdx.edu/bartforge/ziggurat}.

\section{Acknowledgments}

  Thanks much to Jules Kongslie, Mark Jones, Dave Archer,
  Jamey Sharp, Josh Triplett and Bryant York for
  illuminating conversations during the discussion of this
  work.  Thanks also to Jules Kongslie and to James McNames
  and his students for patiently explaining BPF to me and
  answering my questions.  Finally, thanks to Keith Packard
  and Intel for providing the hardware on which this work
  was primarily done.

\bibliographystyle{IEEEbib}
\bibliography{ltrs}

\end{document}

% LocalWords:  Resampling BPF resampling na ive resample pseudocode heapifying
% LocalWords:  resampled
